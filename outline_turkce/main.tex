\documentclass{article}

\usepackage{enumitem}
\usepackage{hyperref}

\begin{document}

\title{Ders İçeriği: Regresyon Analizi}
\author{Muhammed Sezer & Şevval Belkıs Dikkaya}
\date{}

\maketitle

\section{Regresyonun Tanımı (15 dakika)}
\begin{enumerate}[label=\alph*)]
\item Regresyonun tanımı
\begin{itemize}
\item Regresyon, bağımlı değişken ile bir veya daha fazla bağımsız değişken arasındaki ilişkiyi kurmak için kullanılan istatistiksel bir yöntemdir.
\item Bu yöntem, bu ilişkiyi temsil eden en iyi uyumlu çizgiyi (veya eğriyi) bulmaya çalışır.
\end{itemize}

\item Regresyonun önemi
\begin{itemize}
\item Regresyon analizi, finans, pazarlama, mühendislik ve sosyal bilimler gibi çeşitli alanlarda önemlidir.
\item Değişkenler arasındaki ilişkiyi anlamaya ve tahmin yapmaya yardımcı olur veya karar verme sürecinde bilgi sağlayabilir.
\end{itemize}

\item Regresyonun kısa tarihi
\begin{itemize}
\item Regresyon kavramı, 19. yüzyılın sonlarında Sir Francis Galton tarafından ilk kez tanıtılmıştır.
\item Bu yöntemi, babaların ve oğullarının boyları arasındaki ilişkiyi incelemek için kullandı.
\item O zamandan beri, regresyon analizi yoğun bir şekilde çalışılmış ve geliştirilmiş ve şimdi çeşitli alanlarda yaygın olarak kullanılmaktadır.
\end{itemize}

\item Regresyon analizi ile çözülebilecek problemlerin türleri
\begin{itemize}
\item Regresyon analizi, hisse senedi fiyatlarını tahmin etmek, satışları tahmin etmek, reklam harcamaları ile satış arasındaki ilişkiyi analiz etmek ve eğitimin gelir üzerindeki etkisini anlamak gibi çeşitli problemleri çözmek için kullanılabilir.
\end{itemize}

\item Çeşitli alanlardaki regresyon uygulamaları
\begin{itemize}
\item Regresyon analizi, finans, pazarlama, sağlık hizmetleri, sosyal bilimler ve mühendislik gibi çeşitli alanlarda geniş uygulamalara sahiptir.
\item Müşteri davranışını anlamak ve tahmin etmek, satış ve gelirleri tahmin etmek, üretim süreçlerini optimize etmek ve çevresel faktörler ile sağlık sonuçları arasındaki ilişkiyi analiz etmek gibi şeyler için kullanılır.
\end{itemize}
\end{enumerate}


